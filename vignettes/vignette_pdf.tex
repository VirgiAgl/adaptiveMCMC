\documentclass{article}

\usepackage{amsmath, amsthm, amssymb}
\usepackage[round]{natbib}

% The line below tells R to use knitr on this.
%\VignetteEngine{knitr::knitr}

\title{OxWaSP Module 1: Adaptive MCMC}
\author{Virginia Aglietti \and Tamar Loach}

\begin{document}

\maketitle

\section{Introduction to Adaptive MCMC - the AM algorithm}

MCMC algorithms allow sampling from complicated, high-dimensional distributions. Choice of the proposal distribution (from which samples are taken in an attempt to approximate sampling from the target distribution $\pi$) determines the ability of the algorithm to explore the parameter space fully and hence draw a good sample. Adaptive MCMC algorithms tackle this challenge by using samples already generated to learn about the target distribution; they push this knowledge back to the choice of proposal distribution iteratively.

This project explores adaptive MCMC algorithms existing in the literature that use covariance estimators to improve convergence to a target distribution supported on a subset of $\mathbb{R}^d$. In this schema we learn about the target distribution $\pi$ through estimation of its correlation structure from the MCMC samples. We use this correlation structure to improve our estimate of the target.

We first implement an adaptive MCMC algorithm AM \citep{haario2001} which is a modification of the random walk Metropolis-Hastings algorithm. In AM the proposal distribution is updated at time $t$ to be a normal distribution centered on the current point $X_{t-1}$ with covariance $C_t(X_0, ..., X_{t-1})$ that depends on the the whole history of the chain. The use of historic states means the resulting chain is non-markovian, and reversibility conditions are not satisfied. Haario et al show that, with a small update to the usual Metropolis-Hastings acceptance probability, the right ergodic properties and correct simulation of the target distribution none the less remain. The probability with which to accept candidate points in the chain becomes:

$$
\alpha(X_{t-1},Y) = \text{min}\left( 1,\frac{\pi(Y)}{\pi(X_{t-1})}\right)
$$

With $C_t$ given by:

$$
C_t = s_d \text{cov}(X_0, ..., X_{t-1}) + s_d\epsilon I_d
$$

Here $\text{cov}()$ is the usual empirical covariance matrix, and the parameter $s_d = \frac{2.4^2}{d}$ \citep{gelman1996}. $\epsilon$ is chosen to be very small compared to the subset of $\mathbb{R}^d$ upon which the target function is supported. The AM algorithm is computationally feasible %Tamar What is the computional compplexity?
due to recursive updating of the covariance matrix on acquisition of each new sample through the relation:

$$
C_{t+1} = \frac{t-1}{t} C_t + \frac{s_d}{t}(t \bar{X}_{t-1}\bar{X}^T_{t-1} - (t+1)\bar{X}_t \bar{X}^T_t + X_tX_t^T + \epsilon I_d)
$$
 with the mean calculated recursively by:
$$
\bar{X}_{t+1} = \frac{t \bar{X}_{t}  + X_{t+1}}{t+1}
$$

Because of the instability of the covariance matrix, to implement the adaptivity we first run the algorithm with no change to the covariance of the proposal distribution. The adaptation starts at a user defined point in time, and until this time the covariance of the proposal is chosen to represent our best knowledge of the target distribution.

\section{An example - testing the AM algorithm}

We now numerically test the AM algorithm. We have used two different target distributions: a correlated Gaussian distribution $N(0,\Sigma)$ and a "banana"-shaped distribution (\citep{roberts2009}) given by:

$$
f_B\left(x_1,...,x_d\right)\propto \exp \left[ -x_1^2/200 - \frac{1}{2} \left(x_2 + Bx_1^2-100B\right)^2 - \frac{1}{2} \left(x_3^2 + x_4^2 + ... + x_d^2\right) \right]
$$

$B > 0$ is the "bananicty" constant (set to 0.1 throughout) and $d$ is the dimension. We have chosen the correlated Guassian distribution as targetting this demonstrates how the use of empirical covariance improves convergence - we learn the target's covariance as we move through steps of the MCMC. The banana-shaped distribution is an additional example with an irregular shape. We use this to test the ability of the markov chain to fully explore the state space with and without adaption. We first run our implementation of the AM algorithm targetting $N(0,\Sigma)$ with %todo what is the correlation structure used?




